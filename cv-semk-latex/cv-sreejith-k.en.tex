\documentclass{resume}

\renewcommand{\categoryfont}{\sc}

%
% set the space used for category titles here:
% use the same value for oddsidemargin and marginparwidth [the latter 
% 		will be reset to account for marginparsep]
% 
\setlength{\oddsidemargin}{1in}
\setlength{\marginparwidth}{1in}
% 
% calculate other dimensions [textwidth and evensidemargin] 
% in function of oddsidemargin and marginparwidth: 
% would be nicer to put in the class file...
%
\addtolength{\marginparwidth}{-\marginparsep}
\setlength{\evensidemargin}{\oddsidemargin}
\setlength{\textwidth}{\paperwidth}
\addtolength{\textwidth}{-2in}
\addtolength{\textwidth}{-2\oddsidemargin}
\addtolength{\textwidth}{\marginparwidth}
\addtolength{\textwidth}{\marginparsep}
%
%
\setlength{\topmargin}{-0.5in}
%
%
\renewcommand{\labelcitem}{$\diamond$}
\renewcommand{\labelitemi}{$\cdot$}
\newcommand{\first}{$1^{\mbox{\scriptsize st}}$\ }
\newcommand{\second}{$2^{\mbox{\scriptsize nd}}$\ }
\newcommand{\third}{$3^{\mbox{\scriptsize rd}}$\ }

\author{Sreejith Kesavan}

% ------------------------ Address ------------------------ %

\address{Staff Engineer\\
  Nutanix, Inc.\\
  Bangalore, KA, India\\
  {\em Ph:} (+91)-99-8684-9928}{
  {\em email:} \mbox{\small\tt sreejithemk@gmail.com}\\
  {\em web:} \mbox{\small\tt foobarnbaz.com}\\
  {\em github:} \mbox{\small\tt github.com/semk}\\
  {\em linkedin:} \mbox{\small\tt linkedin.com/in/sreejithemk}}

\begin{document}
\maketitle

\begin{category}{Summary}
  \citemnobullet Career spanning more than 11 years, built products for the cloud, storage \& virtualization domains. Currently working as a Staff Engineer, leading the design and development of a cloud mobility solution (Nutanix Move) for the hyper-converged infrastructure (HCI) platform at Nutanix, Inc.
\end{category}

% ------------------------ Work Experience ------------------------ %

\begin{category}{Work \\experience}
  \citem{Nutanix, Inc., Bangalore}
  \citemnobullet \textbf{Staff Engineer} \hfill \textbf{Aug 2015 -- present}
  \citemnobullet Working as a Staff Engineer (Technical Lead) for the Cloud Mobility Services R\&D team at Nutanix, Inc.
  \begin{itemize}
  \item Prototyped an agentless \& incremental migration methodololy for migrating Azure virtual machine workloads to on-premise AHV clusters.
  \item Boosted the migration throughput upto 7x on high latency, lossy networks by leveraging KCP, an opensource UDP based reliable transport protocol.
  \item Added protocol-agnostic stream multiplexing capabilities to the data transport pipe.
  \item Designed and developed the Nutanix Move microservice stack \& application life cycle management (one-click upgrade) using Docker container ecosystem.
  \item Built a minimal Docker container host operating system based on Alpine Linux to ship Nutanix Move as a virtual appliance.
  \item Designed and developed a rules engine to programmatically define and predict workload requirements for database migrations (DB Xtract).
  \item Developed a task execution framework in Python with support for pause/resume/cancel operations and progress monitoring.
  \item Lead the design and development FitCheck, a cluster health diagnostics tool.
  \end{itemize}
  \citemnobullet Technology Stack: \textit{Python, Go, Docker, OpenAPI, gRPC, AWS, Azure, ESXi, KVM.}


  \citem{NetApp, Inc., Bangalore}
  \citemnobullet \textbf{Member of Technical Staff} \hfill \textbf{Sep 2012 -- Aug 2015}
  \citemnobullet Worked as lead developer in a team responsible for the development, testing and debugging of SAN \& NAS data migration platforms.
  \begin{itemize}
  \item Developed data collection and reporting modules for the Inventory Collect Tool (ICT) used in 7-Mode Transition Tool (7MTT).
  \item Developed an XSLT based viewer for the inventory data collected by ICT.
  \item Reduced the overall memory footprint of the collector module upto 3x and improved performance upto 2x.
  \item Developed a migration planner using PyQt to generate homogeneous \& heterogeneous SAN migration plans which can be automated.
  \item Built a PyQt based interface for viewing and uploading data data collector bundle to a centralized NetApp filer.
  \end{itemize}
  \citemnobullet Technology Stack: \textit{Python, PyQt.}


  \citem{K7 Computing Private Limited, Chennai} 
  \citemnobullet \textbf{Development Engineer} \hfill \textbf{Feb 2009 -- Aug 2012}
  \citemnobullet Actively involved in the design and development of a cloud platform that can solve the problems of application scalability, availability and fault-tolerance.
  \begin{itemize}
  \item Developed core services and Google App Engine compatible APIs for a PaaS platform.
  \item Developed a FUSE file system for malware analysis, backed by Samba shares.
  \item Developed various modules and components of a virtualization platform (FluidVM) targeted at VPS hosting providers. Supports Xen, OpenVZ and KVM hypervisors.
  \item Developed HyperVM API emulator for the virtualization management platform.
  \item Developed a management console for data center administrators to manage the virtual machines remotely.
  \end{itemize}
  \citemnobullet Technology Stack: \textit{Python, C, Protocol Buffers, FUSE, libvirt, Xen, KVM.}
\end{category}


% ------------------------ Awards & Honours ------------------------ %

 \begin{category}{Awards \&\\ Honours}
  \citem{Engineering SuperHero Award}, {Nutanix, Inc.} \hfill \textbf{2016}
  \citemnobullet The highest engineering accolade at Nutanix, in recognition of outstanding performance
  \citem{Nutanix Hackathon 3.0}, { Nutanix, Inc.} \hfill \textbf{2016}
  \citemnobullet Project Nixify, Nutanix orchestrator using TOSCA, \#1 in Top Ten Projects
 \end{category}


\begin{category}{Technical Skills}
  \citembullet Programming Languages - \textit{Python, Go, C.}
  \citembullet Development Tools - \textit{Docker, Vagrant, Packer.}
  \citembullet Cloud Environments - \textit{Amazon AWS, Microsoft Azure.}
\end{category}

% ------------------------ Education ------------------------ %

\begin{category}{Education}
  \citem{B-Tech in Computer Science \& Engineering} \hfill \textbf{2004 -- 2008}
  \citemnobullet University of Calicut, Jyothi Engineering College, Kerala.
  \citem{Higher Secondary Education in Computer Science} \hfill \textbf{2002 -- 2004}
  \citemnobullet Kshethra Pravesana Memorial Higher Secondary School, Poothotta, Kerala.
\end{category}

% ------------------------ Opensource ------------------------ %

\begin{category}{Open Source Projects}
  \citemnobullet Curated list of major opensource projects. Checkout the GitHub profile for more details \& source code.
  \citembullet Cyclozzo OSE - Opensource Edition of the Cyclozzo Platform as a Service.
  \citembullet EnlargeWeb - Private and public cloud builder and manager. Supports deployments of Ubuntu, AppScale, Hadoop and Hypertable.
  \citembullet Voldemort - Voldemort is a blog-aware static site generator using Jinja2 and Markdown templates. Inspired by Jekyll, a static site generator written in Ruby.
  \citembullet Game of Life - Conway's Game of Life simuation in Python \& PyQt. Supports LIF, JSON \&
  RLE Cellular Automata file formats.
  \citembullet RegexMate - A visual regular expression test/evaluation tool written in PyQt.
  \citembullet Riak: Various fixes and feature additions to
  the Riak Python Client.
  \citembullet wsnotifier - Lightweight gevent based asynchronous websocket server with HTTP forwarder APIs.
  \citembullet Tuxpaint - A magic tool for Tuxpaint. Can be used as a reference to the magic tool API.
  \citembullet Pytt - A simple BitTorrent tracker written in Python.
  \citembullet GitFS - Helps to use GitHub storage space as a filesystem (FUSE) under Linux.
\end{category}

% ------------------------ Publications ------------------------ %

\begin{category}{Published Articles}
  \citem{Linux For You}, {\em ``The New Scheduler on the Block, Dedicated to Desktops''}. \hfill \textbf{Oct 2009}
  \citemnobullet An article about Brain Fuck Scheduler (BFS) for Linux written by famous Linux Kernel hacker Con Kolivas.
\end{category}

\end{document}
